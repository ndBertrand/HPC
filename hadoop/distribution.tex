\section{Les distributions de Hadoop}
Beaucoup de communautés informatiques se sont données pour mission d’améliorer et d’enrichir les outils et les services Hadoop en développant des modules open source. Il existe cependant de plus en plus de versions commerciales distribuées par des vendeurs de logiciels. Ces versions payantes proposent une version Hadoop personnalisé.  
Parmi les distributions les plus populaires, on distingue quatre grands acteurs qui proposent des services de formation et un support commercial :  
\begin{itemize}
    \item \emph{Cloudera} : Cloudera est non seulement la première distribution d’Hadoop mais également la plus utilisée. il intègre les packages classique mais l’un de ses principaux modules est le Cloudera Impala. Il s’agit d’un module de langage de requête compatible avec Hadoop. Impala permet de gérer une analyse interactive et en temps réel. 
    \item \emph{Hortonworks} : il présente l’avantage d’être une solution rapide rentable et efficace. Il fournit plusieurs services de gestion, surveillance et d’intégration des données. 
    \item \emph{MapR} : MapR est une distribution commerciale d’Hadoop destinée à l’entreprise. Il a été amélioré pour offrir une meilleure fiabilité, la performance et la facilité de stockage Big Data. 
    \item IBM bigInsights :  a été conçu pour facilité son utlisation dans les systèmes d’inforamtions  d’entreprise. 
    \item \emph{Pivotal HD} : fournit une base de données avancée, des services avec plusieurs composants, y compris ses propres bases de données relationnelles parallèles.
\end{itemize}
\cite{oussous_big_2018}







